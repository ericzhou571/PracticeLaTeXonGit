\documentclass[a4paper,12pt]{article}
%   Document Standards
    \usepackage[utf8]{inputenc}
    \usepackage[T1]{fontenc}
    \usepackage{lmodern}

% German language support
	 \usepackage[ngerman]{babel}

%   Mathematics
    \usepackage{amsmath}
    \usepackage{amssymb}
    \usepackage{amsthm}
    \usepackage{amsfonts}

%   Formatting
    \usepackage{graphicx}   % Graphics
    \usepackage{booktabs}   % Tables
    \usepackage{fancyhdr}   % Headers and footers
    \usepackage{caption}
    \usepackage{subcaption}
   	\usepackage[left]{eurosym}
    \usepackage{enumerate}
    \usepackage{tabularx}
    \usepackage{array}
    \usepackage{xcolor}
    \usepackage{rotating}
    \usepackage{multirow}
    \usepackage{natbib}

    % Margins
	\usepackage[a4paper, left=3cm, right=3cm, top=4cm, textheight=23cm]{geometry}	

    % Chapter pages
        \pagestyle{headings}
%        \pagestyle{fancy}
%        \lhead{}
%        \chead{}
%        \rhead{\thepage}
%        \lfoot{}
%        \cfoot{}
%        \rfoot{}
%        \renewcommand{\headrulewidth}{0pt}
%        \renewcommand{\footrulewidth}{0pt}

    % Chapter title pages
        \fancypagestyle{plain}{
        \fancyhf{}
        \fancyhead[L]{\emph{Titel der Arbeit}}
        \fancyhead[R]{\thepage}
        \fancyfoot[L]{}
        \fancyfoot[R]{}
        \renewcommand{\headrulewidth}{0pt}
        \renewcommand{\footrulewidth}{0pt}}
        \renewcommand{\baselinestretch}{1.3}
				\let\footnoteOld\footnote		% Zeilenabstand in der Fußnote wird zurückgesetzt
				\renewcommand{\footnote}[1]{\linespread{1.0}\footnoteOld{#1}\linespread{1.2}}		% Zeilenabstand in der Fußnote wird gesetzt

\begin{document}

\begin{titlepage}
\topskip0cm
\begin{center}
{\Large Karlsruher Institut für Technologie\\[0.4cm]
Institut f\"{u}r Finanzwirtschaft, Banken und Versicherungen\\[0.3cm]
Lehrstuhl Financial Engineering und Derivate\\[0.3cm]
Prof. Dr. Marliese Uhrig-Homburg}\\[3.5cm]
{\large Seminararbeit}\\[0.3cm]
{\large Seminarthema}\\[0.3cm]
{\large WS 20xx/20xx}\\[0.7cm]
{\large Thema 123} \\[0.3cm]
{\Huge Titel des Themas}\\[6cm]
\end{center}
\renewcommand{\baselinestretch}{1.2}\small\normalsize
\begin{tabular}{ll}
        Verfasser:  & Vorname Name\\
                    & Straße Hausnr.\\
                    & PLZ Ort\\
					& E-Mail: E-Mail-Adresse\\\\
        \multicolumn{2}{l}{Karlsruhe, den xx. September 20xx}
    \end{tabular}
    \vfill
\end{titlepage}

\setcounter{page}{1}\renewcommand{\thepage}{\roman{page}}%
\tableofcontents %
\newpage
% \listoffigures \addcontentsline{toc}{chapter}{List of Figures}% in Englisch
\listoffigures \addcontentsline{toc}{section}{Abbildungsverzeichnis} % in Deutsch
% \listoftables \addcontentsline{toc}{chapter}{List of Tables}% in Englisch
\newpage
\listoftables \addcontentsline{toc}{section}{Tabellenverzeichnis} % in Deutsch
\newpage
\setcounter{page}{1}\renewcommand{\thepage}{\arabic{page}}

\section{Einleitung\label{sec:einleitung}}

Nach \citet{Hull2000} gilt...  % natbib verwendet citet statt cite

\section{Überschrift Kapitel 2\label{sec:kap2}}

Hier folgt Kapitel 2.\footnote{Vgl. \citet{Cox1979}, S. 230.}

\subsection{Überschrift Unterabschnitt 2.1\label{sec:ab21}}

\subsection{Überschrift Unterabschnitt 2.2\label{sec:ab22}}

\section{Überschrift Kapitel 3\label{sec:kap3}}

Hier folgt Kapitel 3.\footnote{Vgl. \citet{Skantze2000}, S. 22.}

\newpage

\section{Zusammenfassung und Ausblick\label{sec:outlook}}

%% Apendix
%\addcontentsline{toc}{section}{Bibliography} \nocite{*}
\newpage
\addcontentsline{toc}{section}{\bibname}  % Deutsch
% \nocite{*} % Zeige nicht-zitierte Literatur im Literaturverzeichnis

\bibliographystyle{apalike_ger_manipulation}
\bibliography{Seminar_Bib}
\newpage

\appendix
\section{Überschrift Anhang A\label{sec:anhangA}}
\newpage
\section{Überschrift Anhang B\label{sec:anhangB}}
\newpage \thispagestyle{empty}

\end{document}
